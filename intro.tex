Traditionally, in drug discovery, target identification to develop novel agents for any given disease is carried out on a case-by-case basis. In this model, individual scientists act as project champions for targets based on available literature and local expertise. Nevertheless   drug discovery is extremely costly and failure-prone \cite{ferrero2017}. Nevertheless:

\begin{enumerate}
    
    \item With increasing publication rates, it is becoming impossible to maintain an overview over an increasingly vast scientific literature. The large quantity not just biomedical research literature but also orthogonal perspectives continually being produced renders it impossible for individual researchers and corporations to keep up to date \cite{brown2018}. Therefore, the development of \textbf{alert systems for emerging targets and trends in the literature} at a genome scale is of prime importance in the pharmaceutical sector.
    
    \item In recent years a wealth of biological data from multiple domains has become available in \textbf{public data repositories} \cite{brown2018}. A firm integration of this extensive and extremely valuable corpus of scientific literature is necessary for a rapid generation of new, high-quality hypothesis for drug discovery without the need of human intervention \cite{ferrero2017, brown2018}.
    
    \item Drug discovery is extremely costly and failure-prone \cite{ferrero2017}. As the cost of successful drug development continues to increase, the productivity of the industry as a whole in launching new drugs remains flat \cite{brown2018}.

\end{enumerate}

\noindent
Hence, both putative target and topic prioritisation in drug discovery (i) is of paramount importance to maximise the chances of success in clinic \cite{ferrero2017} and (ii) ensures a sustainable business in the long term \cite{ferrero2017}.

\subsection{Proposed approach}
Based on the \href{https://goo.gl/QrfdfY}{initial proposal}, this PhD iCASE will focus on the development of an innovative integrated resource which uses machine learning algorithms to exploit a newly generated semantic repository to facilitate a systematic genome scale ranking of potential targets in the areas of type 2 diabetes, non-alcoholic steatohepatitis (NASH) and metabolic syndrome: putative therapeutic target prioritisation (PTTP).

In order to do so, the student divided the project into two subparts:
\begin{enumerate}
\item Develop machine learning methods to create an alert system for emerging targets from PubMed 2019 version
\item Develop a comprehension about the publicly available biomedical resources to create an integrated resource for putative therapeutic target prioritisation (PTTP) for the next months of the project (please see Project Description: Machine Learning for drug target identification also presented to this Tribunal Committee)
\end{enumerate}
\newpage