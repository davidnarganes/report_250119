\subsection{DISEASES}
\label{subsec:DISEASES}
\href{https://goo.gl/KGQsd9}{DISEASES} integrates evidence on GDAs from (i) automatic text mining, (ii) manually curated literature, (iii) cancer mutation data, and (iv) GWAS \cite{DISEASES2015}. They further integrate all evidence data and define confidence scores that facilitate comparison of the different types and sources of evidence \cite{DISEASES2015}.

They use a Named Entity Recogniser (NER) that maps diseases in Medline abstracts to the \href{http://disease-ontology.org/}{Disease Ontology} \cite{schriml2012} in two processes: (i) namely recognition and (ii) normalization (also known as identification, mapping, or grounding).

DISEASES integrates the GDAs extracted from automatic text mining with evidence from databases with permissive licenses, namely manually curated associations (see Table \ref{tab:diseases_data}).
\begin{table}[H]
\centering
    \begin{tabular}{c|c}
         Source & Description  \\
         \hline
         \href{https://goo.gl/gnQSYt}{GHR} & Genetic variation and diseases at NIH*\\
         \href{https://goo.gl/d2PhpC}{UniProtKB} & Curated protein functional data \cite{uniprot2017} \\
         \href{https://goo.gl/hYjVeR}{DistiLD} & GWAS linked to ICD-10 codes \cite{palleja2011} \\
         \href{https://goo.gl/qcY8qo}{COSMIC} & Somatic mutation cancer data \cite{cosmic2017}
    \end{tabular}
    \caption{Data sources used by DISEASES \cite{DISEASES2015} \label{tab:diseases_data}}
\end{table}